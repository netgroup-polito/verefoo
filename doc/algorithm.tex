As mentioned previously, VeriGraph needs to set up the network model where reachability properties are verified by means of Z3. In order to perform this task, we need to prepare the verification environment through the \texttt{reachabilitySrc2Dst} function. This function performs the reachability test between the source and destination nodes and receives as input:
\begin{itemize}
	\item \texttt{GRAPH\_PATH} indicates the path of the JSON file where is described the NF-FG
	\item \texttt{SRC} is the name of the source node of which we want to verify the reachability to the DST node
	\item \texttt{DST} is the destination node name of the reachability test
	\item \texttt{CONFIGURATION\_PATH} is the path where the VNF configurations are stored
\end{itemize}
The pseudo-code of this function is the following:
\begin{algorithm}
	\label{alg1}
	\caption{VNF chains retrieval from Neo4J database algorithm and Verification process initialization}
	\begin{algorithmic}[1]
		\Procedure{reachabilitySrc2Dst(String graph\_path, String src, String dst, String configuration\_path)} {}
		\State List$<$Node[]$>$ chain$\_$list = Neo4JManager.getPathFromNeo4J(graph\_path, src, dst)
		\State Map VNF\_configuration = retrieveConfiguration(configuration\_path)
		\For{each chain $c$ in chain$\_$list }
		\State resetZ3State()
		\State test = createReachTest($c$, VNF\_configuration)
		\State test.check.CheckReachabilityProperty(src, dst)
		\EndFor
		\EndProcedure
	\end{algorithmic}
\end{algorithm}

The \texttt{createReachTest} function generates the formula for modelling the VNF chain where the reachability verification is performed. This function needs as input:
\begin{itemize}
	\item \texttt{CHAIN}, which is the list of chains extracted from the VNF graph thanks to No4JManager;
	\item \texttt{VNF\_CONFIGURATION} collects the configurations of the network functions in the VNF catalog.
\end{itemize}
The semantic of the configuration parameters passed to VeriGraph depends on the VNF type. Having described the models of the VNFs supported by VeriGraph and how to configure them in Section~\ref{VNFmodels}, we briefly recap what we expect the \texttt{VNF\_CONFIGURATION} input contains:
\begin{itemize}
	\item \textbf{NAT}: a set of private addresses that represent the hosts in the internal network; 
	\item \textbf{Web Cache}: the list of network nodes that belong to the internal network;
	\item \textbf{Anti-spam}: the set of blacklist email addresses;
	\item \textbf{ACL and Leaning firewalls}: a set of \textit{$<source, destination>$} pair of addresses, which are allowed (Learning firewall) or not (ACL firewall) to communicate between each other;
	\item \textbf{End-host, Mail Client/Server, Web CLient/Server}: at this version of VeriGraph, the tool does not support end-host configurations. In other words, we cannot specify which traffic flow end-hosts send without changing their models (e.g., a client can generate packet with specific port number, destination address etc.). It could be useful to extend end-host model in this direction.
\end{itemize}
 VeriGraph needs also to know how the traffic must be forwarded in a VNF chain. This information is configured through the \texttt{routingTable} function, which initializes the routing table of each node in the VNF chain. Here, the function receives as input:
 \begin{itemize}
 	\item \texttt{CHAIN} is the list of network nodes that compose the chain
 	\item \texttt{NETWORK} is the object that represent the VNF chain of which the network nodes, the addresses, and more have been initialized in the previous step 
 \end{itemize}
\begin{algorithm}[H]
	\caption{Reachability test creation}
	\begin{algorithmic}[1]
		\Procedure{createReachTest(Node[] chain, Map VNF\_Configuration)} {}
		\State Map nodes
		\State Context context
		\State Network network
		\For{each node $n$ in chain }
			\State nodes.put(n.name, n.address)
		\EndFor
		\State context = initializeContext(nodes.getKeys(), nodes.getValues()) 
		\State network = initializeNetwork(context) 
		\State network.attach(nodes.getKeys())
		\For{each node $n$ in nodes}
			\State network.setAddressMapping(n.name, n.address)
			\If{n.functional\_type == ``nat"} 
				\State{nat = initializeNAT(n.name, context, network)}
				\State{nat.configure(VNF\_configuration[nat])} 
			\ElsIf{n.functional\_type == ``fw"} 
				\State{fw = initializeFW(n.name, context, network)}
				\State{fw.configure(VNF\_configuration[fw])} 
			\ElsIf{n.functional\_type == ``spam"}
				\State{...}
			\Else {``Function Type out of catalog"}
			\EndIf
		\EndFor
										
%		\algstore{myalg}
%	\end{algorithmic}
%\end{algorithm}
%
%\begin{algorithm}                     
%	\begin{algorithmic} [1]                   % enter the algorithmic environment
%		\algrestore{myalg}
		\State routingTable(chain,network)
		\EndProcedure
	\end{algorithmic}
\end{algorithm}

Here we use the \texttt{routingTable(string node\_name, List<String, String> entries)} function exposed by the Network object that needs as inputs:
\begin{itemize}
	\item \texttt{node\_name} is the name of the network node where we are configuring the routing table
	\item \texttt{List<String, String> entries} is a list of String name pair, where the first name indicates the node name towards packet is delivered, while the second name is the next hop in the chain
\end{itemize}
\begin{algorithm}[H]
	\caption{Routing table configuration}
	\begin{algorithmic}[1]
		\Procedure{routingTable(Node[] chain, Network network)} {}
		\For{$i$ = 0; $i < chain.length$; $i++$ }
			\For{$j$ = 0; $j < chain.length$; $j++$ }
				\If{$i$ != $j$} 
					\If{$i < j$} 
						\State{network.routingTable(chain[$i$].name,\{chain[$j$].address, chain[$i+1$].address\})}
					\ElsIf{$i > j$} 
						\State{network.routingTable(chain[$i$], \{chain[$j$].address, chain[$i-1$].address\})}
					\EndIf
				\EndIf
			\EndFor
		\EndFor
		\EndProcedure
	\end{algorithmic}
\end{algorithm}
